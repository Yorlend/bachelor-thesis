\chapter{Конструкторский раздел}

В данном разделе разработан метод опорных точек сигнала для геопозиционирования в помещениях с помощью Wi-Fi. Описаны особенности предлагаемого метода, ключевые этапы и требования к формату входных и выходных данных для работы метода. В конце раздела описаны структуры данных, используемые в алгоритмах.

\section{Функциональная модель метода}

\section{Структуры данных}

\section{Алгоритм вычисления...}

\section{Требования к ПО}

\section{Ограничения на входные данные}

\section{Архитектура ПО}

Программное обеспечение, разрабатываемое в работе, представлено в виде клиент-серверного приложения. Серверная часть разрабатывается независимо от клиентской, предоставляя API для вызовов необходимых пользователю функций. Клиентская часть представлена в виде SPA-приложения. (дополнить, причесать)

...
