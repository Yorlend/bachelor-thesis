\begin{essay}{}

    Целью работы является разработка и реализация метода опорных точек сигнала для геопозиционирования в помещении с помощью WiFi.

    Выпускная квалификационная работа содержит четыре раздела. В первом разделе проводится анализ существующих методов геопозиционирования в помещении, а также приводится классификация методов геопозиционирования с помощью WiFi. Во втором разделе описан реализуемый метод, приведены схемы алгоритмов, используемых в методе. В третьем разделе приведены реализации алгоритмов, используемых для геопозиционирования в помещении с помощью WiFi. В четвертом разделе ...

    Новизна работы заключается в применении градиентов уровней сигналов в сочетании с опорными точками сигналов. Представленный в работе подход позволяет увеличить точность определения местоположения объекта в помещении.

    Ключевые слова: метод опорных точек сигнала, геопозиционирование, WiFi, градиент.

\end{essay}