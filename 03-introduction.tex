\chapter*{ВВЕДЕНИЕ}
\addcontentsline{toc}{chapter}{ВВЕДЕНИЕ}

Точное определение местоположения внутри помещений является задачей, имеющей множество применений в областях, таких как управление зданиями, отслеживание ресурсов, навигация, а также обеспечение безопасности и аварийно-спасательных операций \cite{basebook}. В традиционных системах позиционирования, основанных на использовании глобальных навигационных спутниковых систем (ГНСС), такая задача становится нерешаемой из-за ослабления или отсутствия сигнала ГНСС в закрытых помещениях \cite{GPSdrawbacks}.

В работе предлагается метод опорных точек сигнала для геопозиционирования внутри помещений с использованием беспроводных локальных сетей (Wi-Fi). Данный метод основан на пространственном картировании характеристик радиосигналов от точек доступа Wi-Fi и их последующем использовании для определения местоположения. Ключевым преимуществом этого подхода является возможность использования существующей Wi-Fi инфраструктуры без необходимости установки дополнительного оборудования.

Целью работы является разработка и исследование метода определения местоположения в помещениях на основе характеристик Wi-Fi сигналов. Для достижения этой цели необходимо решить следующие задачи:

\begin{enumerate}
    \item Провести анализ существующих методов геопозиционирования в помещениях с использованием Wi-Fi сигналов.
    \item Разработать алгоритм определение местоположения объекта внутри помещения на основе сравнения характеристик сигналов агента с пространственной моделью.
    \item Реализовать алгоритм определения местоположения объекта внутри помещения на основе сравнения характеристик сигналов агента с пространственной моделью.
    \item Провести оценку точности предложенного метода.
\end{enumerate}

Решение поставленных задач позволит создать эффективный метод геопозиционирования в помещениях, не требующий установки специального оборудования и основанный на использовании существующей Wi-Fi инфраструктуры. Результаты работы могут найти применение в различных областях, связанных с отслеживанием местоположения людей и объектов внутри зданий.
