\chapter*{ЗАКЛЮЧЕНИЕ}
\addcontentsline{toc}{chapter}{ЗАКЛЮЧЕНИЕ}

В ходе работы была проанализирована предметная область, существующие методы и полходы к решению задачи геопозиционирования в помещении. Была формализована задача геопозиционирования в помещении методом опорных точек с помощью WiFi.

Был спроектирован метод опорных точек сигнала для геопозиционирования в помещениях с помощью WiFi. Описаны основные алгоритмы и структуры данных, используемые при реализации. На основе описанной документации был разработан программный продукт, позволяющий моделировать геопозиционирование в помещении методом опорных точек сигнала с использованием сил сигналов WiFi.

Было проведено исследование точности разработанного метода геопозиционирования в помещении. В результате исследования был сделан вывод об эффективности разработанного метода в случае равномерной и оптимизированной расстановке опорных точек.
