\chapter{Исследовательский раздел}

\section{Исследование точности метода}

Целью исследования является оценка точности работы реализованного метода, а также сравнение с точностью методов KNN и WKNN. Исследование будет проводиться с использованием набора данных, содержащего в себе топологию помещения, позиции точек доступа и наборов координат и уровней сигналов опорных точек. 

\subsection{Постановка исследования}

Исследование проводится на наборе данных, полученном в помещении, в котором находится три роутера \ref{img:room}.

\includeimage
    {room}
    {f}
    {H}
    {0.7\linewidth}
    {Помещение для исследования}

Набор данных для исследования содержит:
\begin{itemize}[label=---]
    \item измерения RSS в 24 опорных точках, полученные в результате усреднения измерений в течение 2-х минутного периода с целью устранения резких флуктуаций сигнала.
    \item показания RSS полученные в результате интерполяции исходных данных с помощью регрессии на основе гауссовских процессов, содержащие 204 точки.
\end{itemize}

Все данные о мощности сигнала записаны в децибелах относительно 1 милливатта (dBm), в наборе данных все значения RSS являются отрицательными числами, что соответствует ослаблению сигнала.

\subsection{Результат исследования}

Для оценки точности геопозиционирования будет построена выборочная функция распределения абсолютной ошибки.

\begin{enumerate}
    \item Для каждой точки с известными уровнями сигналов вычисляется расстояние между ее истинной и оцененной координатами, полученными с помощью реализованного метода, а также методов KNN и WKNN.
    \item Полученные значения абсолютных ошибок геопозиционирования сортируются в порядке возрастания.
    \item Для каждого значения ошибки вычисляется соответствующая вероятность (частота) как отношение количества ошибок, меньших или равных данному значению, к общему числу ошибок.
    \item Полученные пары <<ошибка-вероятность>> наносятся на график, представляющий собой выборочную функцию распределения абсолютной ошибки геопозиционирования.
\end{enumerate}

\includeimage
    {acc_cdf}
    {f}
    {H}
    {\linewidth}
    {График выборочной функции распределения абсолютной ошибки}

\subsection{Выводы}

В результате исследования были получены следующие результаты:

\begin{enumerate}
    \item Точность геопозиционирования в помещении с помощью разработанного метода в 90\% случаев не превосходит 6 метров.
    \item Точность геопозиционирования в помещении с помощью KNN в 90\% случаев не превосходит 17 метров.
    \item Точность геопозиционирования в помещении с помощью WKNN в 90\% случаев не превосходит 14 метров.
\end{enumerate}

Разработанный метод оказался точнее методов KNN и WKNN на наборе данных, на котором проводилось исследование.

\clearpage

\section{Исследование влияния положения агента на точность}

Целью данного исследования является изучение влияния положения агента на точность геопозиционирования в помещении с использованием сигналов WiFi. Исследование укажет на факторы, влияющие на работу систем позиционирования в помещении.

\subsection{Постановка исследования}

Для проведения исследования была собрана следующая конфигурация роутеров и опорных точек.

\includeimage
    {map_setup}
    {f}
    {H}
    {0.7\linewidth}
    {Конфигурация роутеров и опорных точек}

Для проведения исследования необходимы следующие данные.

\begin{enumerate}
    \item Набор точек доступа, координаты которых известны.
    \item Набор опорных точек, координаты которых известны.
    \item Набор тестовых точек, координаты которых известны, в которых будет производится геопозиционирование.
\end{enumerate}

Представленные данные обрабатываются следующим образом.
\begin{enumerate}
    \item Для каждой опорной точки измеряется вектор RSSI от доступных точек доступа.
    \item Для каждой тестовой точки измеряется вектор RSSI от доступных точек доступа.
    \item Вектор RSSI от каждой тестовой точки подается на вход методу геопозиционирования, вычисляется абсолютная ошибка между истинной и предсказанной координатами.
    \item Создается карта ошибок, отображающая абсолютную ошибку для каждого региона в исследуемом пространстве. Регион представляет из себя квадратную область со стороной 0.5 м.
\end{enumerate}

\subsubsection{Моделирование вектора уровней сигналов}

Модель распространения сигнала, описываемая формулой (\ref{eq:mod}), основана на концепции затухания сигнала в зависимости от расстояния и широко используется в сфере беспроводных технологий~\cite{propagation}.

\begin{equation}
    RSSI = -30 \cdot \lg(d),
    \label{eq:mod}
\end{equation}
где $d$ --- расстояние между точкой доступа и тестовой точкой, в которой вычисляется сила сигнала.

Согласно этой модели, значение $RSSI$ является логарифмической функцией расстояния $d$ между передатчиком и приемником. Коэффициент $-30$ в формуле отражает скорость, с которой сигнал ослабевает при увеличении расстояния.

\clearpage

\subsection{Результат исследования}

На рисунках \ref{img:knn_scatter}, \ref{img:wknn_scatter} и \ref{img:gradient_scatter} представлены карты ошибок методов геопозиционирования KNN, WKNN и разработанного метода соответственно.

\includeimage
    {knn_scatter}
    {f}
    {H}
    {\linewidth}
    {Карта ошибок метода KNN}

\includeimage
    {wknn_scatter}
    {f}
    {H}
    {\linewidth}
    {Карта ошибок метода WKNN}

\includeimage
    {gradient_scatter}
    {f}
    {H}
    {\linewidth}
    {Карта ошибок разработанного метода}

\subsection{Выводы}

По результатам исследования можно сделать вывод, что разработанный метод предлагает высокую точность геопозиционирования. При приближении агента к роутеру, разработанный метод имеет абсолютную ошибку на уровне 6 метров. Однако при приближении агента к опорным точкам, метод предлагает более высокую точность, ошибка составляет 1-2 метра, в то время как KNN в тех же условиях имеет абсолютную ошибку порядка 10 метров, а WKNN порядка 4 метров.
