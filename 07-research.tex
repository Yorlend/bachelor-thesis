\chapter{Исследовательский раздел}

\section{Исследование точности метода}

Целью исследования является оценка точности работы реализованного метода, а также сравнение с точностью методов KNN и WKNN. Исследование будет проводиться с использованием набора данных, содержащего в себе топологию помещения, позиции точек доступа и наборов координат и уровней сигналов опорных точек. 

\subsection{Постановка исследования}

TODO

\subsection{Результат исследования}

\includeimage
    {acc_cdf}
    {f}
    {H}
    {\linewidth}
    {График выборочной функции распределения абсолютной ошибки}

\includeimage
    {acc_kde}
    {f}
    {H}
    {\linewidth}
    {График ядерной оценки плотности функции распределения абсолютной ошибки}

\section{Исследование влияния положения опорных точек на точность метода}

\section{Исследование влияния количества опорных точек на точность метода}

\section{Исследование влияния положения роутеров на точность метода}

\section{Исследование влияния количества роутеров на точность метода}

\section{Сравнение с аналогами}
