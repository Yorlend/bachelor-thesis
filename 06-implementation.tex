\chapter{Технологический раздел}

\section{Выбор средств реализации}

Для реализации серверной части приложения будет использоваться язык программирования Python~\cite{python} в связи с обширной библиотечной базой языка для построения веб-приложений, а также инструментами для работы с данными, позволяющими выполнять необходимые вычисления~\cite{pyproblems}. Для реализации клиентской части будет использован язык программирования TypeScript~\cite{typescript} в сочетании с фреймворком Vue3. Для организации стилистической части приложения будет использована библиотека компонентов shadcn-vue.

\section{Формат входных и выходных данных}

Входными данными являются:

\begin{itemize}[label=---]
    \item список вершин, представляющий собой топологию комнаты;
    \item координаты опорных точек;
    \item действительная позиция агента;
    \item используемый для геопозиционирования метод;
    \item координаты точек доступа.
\end{itemize}

Выходными данными являются:

\begin{itemize}[label=---]
    \item предсказанная позиция агента;
    \item координаты ближайшей опорной точки.
\end{itemize}

\clearpage

\section{Реализации алгоритмов}

\includelisting
    {method.py}
    {Реализация метода опорных точек сигнала}

\section{Тестирование ПО}

\textbf{TODO}

\section{Сборка и конфигурирование}

\textbf{TODO}
