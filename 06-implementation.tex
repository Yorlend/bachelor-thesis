\chapter{Технологический раздел}

\section{Выбор средств реализации}

Для реализации серверной части приложения будет использоваться язык программирования Python~\cite{python} в связи с обширной библиотечной базой языка для построения веб-приложений, а также инструментами для работы с данными, позволяющими выполнять необходимые вычисления~\cite{pyproblems}. Для реализации клиентской части будет использован язык программирования TypeScript~\cite{typescript} в сочетании с фреймворком Vue3. Для организации стилистической части приложения будет использована библиотека компонентов shadcn-vue.

\section{Формат входных и выходных данных}

Входными данными являются:

\begin{itemize}[label=---]
    \item список вершин, представляющий собой топологию комнаты;
    \item координаты опорных точек;
    \item действительная позиция агента;
    \item используемый для геопозиционирования метод;
    \item координаты точек доступа.
\end{itemize}

Выходными данными являются:

\begin{itemize}[label=---]
    \item предсказанная позиция агента;
    \item координаты ближайшей опорной точки.
\end{itemize}

\clearpage

\section{Реализации алгоритмов}

\includelisting
    {method.py}
    {Реализация метода опорных точек сигнала (часть 1)}

\includelisting
    {method-2.py}
    {Реализация метода опорных точек сигнала (часть 2)}

\clearpage

\includelisting
    {method-3.py}
    {Реализация метода опорных точек сигнала (часть 3)}

\clearpage

\section{Тестирование ПО}

Тестирование проводится в следующих условиях:

\begin{itemize}[label=---]
    \item помещение, описанное списком вершин: $(0, 0), (0, 50), (50, 50), (50, 0)$;
    \item три роутера, положения которых: $(5, 10), (40, 10), (35, 30)$;
    \item опорные точки и агент, координаты которых варьируются.
\end{itemize}

На выходе будет тестироваться предсказание позиции агента в результате работы метода, реализованного в программном продукте. Результаты тестирования представлены в таблице \ref{tab:test}.

\begin{table}[H]
    \caption{Результаты тестирования программного продукта}
    \label{tab:test}
    \centering
    \begin{tabular}{|c|c|c|}
        \hline
        Координаты опорных точек & Координаты агента & Выходные данные \\
        \hline
        $(15, 19), (30, 15), (30, 30)$ & $(23, 20)$ & $(22, 19)$ \\
        \hline
        $(15, 19), (30, 15), (30, 30)$ & $(10, 10)$ & $(5, 17)$ \\
        \hline
        $(15, 19), (30, 15), (30, 30)$ & $(19, 30)$ & $(20, 23)$ \\
        \hline
        $(15, 19), (30, 15), (30, 30), (15, 35)$ & $(19, 30)$ & $(18, 30)$ \\
        \hline
    \end{tabular}
\end{table}

По результатам тестирования можно сделать вывод, что метод опорных точек сигнала аппроксимирует положение агента с точностью от 0.5 до 4 метров.
