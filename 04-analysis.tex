\chapter{Аналитический раздел}

\section{Предметная область}

Геопозиционирование в помещениях --- это комплекс технологий, методов и алгоритмов, предназначенных для определения местоположения объектов (людей, устройств, товаров и т.д.) внутри закрытых пространств~\cite{basebook}. Данная предметная область находится на стыке навигационных систем, беспроводных сетей, инерциальной навигации, компьютерного зрения и обработки пространственных данных. В отличие от высокоточного позиционирования на открытом пространстве с использованием ГНСС, системы геопозиционирования в помещениях вынуждены использовать альтернативные технологические решения, учитывающие особенности закрытых пространств~\cite{vislight}.

Потребность в точном и достоверном определении местоположения объектов внутри зданий обусловлена широким кругом практических задач:

\begin{itemize}[label=---]
    \item навигация и ориентирование людей внутри зданий (торговые центры, аэропорты, вокзалы, медицинские учреждения и т.д.)~\cite{intro};
    \item отслеживание перемещения персонала, сотрудников, пациентов в офисах, больницах, промышленных объектах~\cite{staffpos};
    \item управление перемещением мобильных роботов, беспилотных тележек и другой автономной техники в логистических центрах, на складах, на производственных площадках~\cite{basebook};
    \item контроль доступа и безопасности в зданиях~\cite{accesscontrol};
    \item обнаружение местоположения вызывающих тревогу объектов (пожар, нарушители и т.п.)~\cite{trespassers};
    \item аналитика посещаемости, поведения посетителей в торговых центрах, музеях, выставках~\cite{occupancy};
\end{itemize}

Исходя из разнообразия и важности прикладных задач, можно утверждать, что технологии геопозиционирования в помещениях играют ключевую роль в развитии <<умных>> городов, зданий, логистики и других перспективных направлений цифровой трансформации.

\section{Подходы работы систем геопозиционирования в помещении}

\subsection{Особенности и ограничения геопозиционирования в помещениях}

Геопозиционирование в помещениях сталкивается с целым рядом технических сложностей, обусловленных спецификой закрытых пространств~\cite{basebook}. Если на открытом воздухе позиционирование базируется на использовании сигналов глобальных навигационных спутниковых систем (ГНСС), то в закрытых помещениях данный подход не подходит из-за высокой погрешности позиционирования~\cite{intro}. Основные ограничения и проблемы геопозиционирования в помещениях включают:

\begin{enumerate}
    \item Экранирование ГНСС-сигналов --- конструкции зданий, стены, перекрытия, металлические элементы интерьера существенно ослабляют и искажают спутниковые сигналы, делая их непригодными для высокоточного позиционирования внутри помещений~\cite{shielding}. Переотражение сигналов от различных объектов также вносит значительные ошибки в определение местоположения~\cite{shielding}.
    \item Сложность и неоднородность радиопокрытия --- в отличие от открытых пространств, где радиопокрытие сотовых сетей, WiFi, Bluetooth относительно равномерно, в помещениях наблюдаются значительные перепады уровня сигналов, обусловленные расположением стен, мебели, оборудования, что затрудняет калибровку и предсказание распространения радиоволн~\cite{heterogenity}.
    \item Необходимость установки дополнительной инфраструктуры --- для реализации точного геопозиционирования в помещениях может потребоваться развертывание специальных навигационных маяков, ретрансляторов, антенн, датчиков, серверов, что обусловливает значительные капитальные и эксплуатационные расходы~\cite{basebook}.
    \item Динамичность и изменчивость обстановки --- помещения не являются статичной средой --- они постоянно меняются (перестановка мебели, открытие/закрытие дверей, наличие людей и т.д.), что ведет к искажению радиосигналов и требует непрерывной адаптации и калибровки системы геопозиционирования~\cite{adaptation}.
    \item Ограничения по энергопотреблению и вычислительной мощности --- мобильные устройства, используемые для геопозиционирования в помещениях, имеют существенные ограничения по автономности, вычислительным ресурсам и тепловыделению, что накладывает жесткие требования к алгоритмам и аппаратным решениям~\cite{econsumption}.
    \item Приватность и защита данных --- геопозиционирование сопряжено с проблемами конфиденциальности, требующими соблюдения этических принципов, законодательных норм и политик безопасности~\cite{privacy}.
\end{enumerate}

Ключевыми технологическими направлениями в области геопозиционирования являются:

\begin{enumerate}
    \item Использование альтернативных радиотехнологий~\cite{altradio}.
    \item Применение инерциальных измерительных систем~\cite{inertialsys}.
    \item Методы компьютерного зрения и SLAM~\cite{compvision}.
    \item Интеграция различных сенсоров и данных~\cite{multsensors}.
    \item Облачные вычисления и машинное обучение~\cite{cloudML}.
\end{enumerate}

\subsection{Использование альтернативных радиотехнологий}

Поскольку ГНСС не подходят для позиционирования в закрытых помещениях, широкое применение находят альтернативные радиотехнологии, использующие разнообразные беспроводные сети и сигналы.

\begin{enumerate}
    \item Wi-Fi~\cite{wifi}. Благодаря массовому распространению точек доступа WiFi в помещениях, данная технология широко применяется для геопозиционирования в помещениях~\cite{trends}. Измерение уровней сигналов (RSS) от нескольких точек доступа позволяет оценить местоположение устройств с точностью до 2-3 метров. Более сложные методы, такие как метод опорных точек сигнала~\cite{fingerprint} или измерение времен распространения~\cite{toa}, способны обеспечить точность до 1 метра.
    \item Bluetooth~\cite{bluetooth}. Спецификация Bluetooth 5.1 добавила поддержку определения направления, что в совокупности с измерением расстояний обеспечивает позиционирование с точностью до 10 см~\cite{basebook}. Использование сети маяков Bluetooth Low Energy (BLE) является одним из наиболее популярных решений в области геопозиционирования в помещениях.
    \item UWB (сверхширокополосные системы)~\cite{uwb}. Технология UWB, использующая широкополосные радиоимпульсы, обладает высокой точностью определения расстояний (до 10 см) и может применяться для трекинга объектов в помещениях с точностью до 30 см~\cite{uwbaccuracy}.
    \item RFID~\cite{rfid}. Системы радиочастотной идентификации позволяют определять местоположение агента в пределах зоны обнаружения антенны с точностью до 1 метра~\cite{basebook}. Данная технология широко используется для отслеживания перемещения товаров, активов на складах и в цепочках поставок~\cite{shielding}.
\end{enumerate}

Комбинирование данных технологий, использование множества приемо-передающих устройств, а также методы машинного обучения и интеграции с другими сенсорами позволяют создавать высокоточные системы геопозиционирования в помещении, обеспечивающие точность до 10-20 см~\cite{basebook}.

\subsection{Применение инерциальных измерительных систем}

Для преодоления ограничений, связанных с использованием радиосигналов, активно развиваются решения на основе инерциальных измерительных устройств~\cite{inertial} (ИИУ) --- акселерометров, гироскопов, магнитометров. Данные сенсоры, встроенные в смартфоны, носимые устройства, роботов, позволяют отслеживать перемещения объектов и вычислять их текущие координаты при помощи алгоритмов счисления координат (dead reckoning)~\cite{inertialsys}.

Основные принципы работы инерциальной навигации~\cite{inertialprinciples}:

\begin{itemize}[label=---]
    \item акселерометр измеряет ускорения объекта в 3 осях, которые интегрируются дважды для получения текущих координат;
    \item гироскоп измеряет угловые скорости вращения, которые интегрируются для определения изменения ориентации;
    \item магнитометр измеряет направление магнитного поля Земли, что позволяет вычислять азимут движения.
\end{itemize}

Комбинация данных сенсоров в единой инерциальной измерительной системе (ИИС) дает возможность отслеживать перемещение объекта в пространстве, постоянно вычисляя его местоположение и ориентацию относительно начальной точки.

Однако, у чисто инерциальных решений есть ряд недостатков~\cite{basebook}:

\begin{itemize}[label=---]
    \item накопление ошибок в процессе интегрирования ускорений и угловых скоростей, что приводит к постепенному дрейфу вычисляемых координат;
    \item невозможность определения абсолютного местоположения без привязки к известным опорным точкам;
    \item ограниченность времени работы из-за быстрого расходования энергии батарей.
\end{itemize}

Для преодоления данных проблем применяются гибридные подходы~\cite{hybridilf}, совмещающие инерциальную навигацию с другими технологиями позиционирования - RFID, WiFi, BLE, компьютерным зрением. Сочетание разнородной сенсорной информации и использование методов фильтрации, оценки состояния и машинного обучения позволяет обеспечить высокоточное и надежное геопозиционирование в помещениях~\cite{basebook}.

\subsection{Методы компьютерного зрения и SLAM}

Альтернативным направлением геопозиционирования в помещениях являются технологии, основанные на компьютерном зрении и одновременной локализации и картографировании (SLAM - Simultaneous Localization And Mapping)~\cite{slambased}.

В SLAM-подходе используются камеры, установленные на мобильных устройствах (смартфоны, роботы), которые захватывают изображения окружающей обстановки. Специальные алгоритмы компьютерного зрения анализируют видеопоток, распознают визуальные ориентиры (текстуры, геометрические формы, QR-коды и т.д.), отслеживают их положение и движение. Совмещая эту информацию с данными инерциальных сенсоров, можно вычислять текущие координаты объекта в пространстве.

Технология SLAM не только определяет местоположение, но и создает трехмерную карту окружающей среды, используя данные камер и ИИУ. Таким образом достигается одновременная локализация объекта и построение пространственной модели помещения.

Основные преимущества решений на базе компьютерного зрения~\cite{compvision}:

\begin{itemize}[label=---]
    \item отсутствие необходимости в развертывании специальной инфраструктуры (маяков, точек доступа) --- достаточно камер мобильных устройств;
    \item возможность одновременного отслеживания нескольких объектов в поле зрения;
    \item высокая точность позиционирования (до 10 см) в определенных условиях;
    \item синергия с технологиями дополненной реальности.
\end{itemize}


Вместе с тем, технологии компьютерного зрения и SLAM имеют ряд ограничений~\cite{slambased}:

\begin{itemize}[label=---]
    \item высокие вычислительные и энергетические требования;
    \item чувствительность к освещению, видимости, загроможденности среды;
    \item необходимость предварительного картографирования помещений;
    \item сложность масштабирования на большие пространства.
\end{itemize}

\subsection{Интеграция различных сенсоров и данных}

Использование комбинации различных сенсоров, включая Wi-Fi, Bluetooth, RFID, ультразвук, видеокамеры и инерциальные измерительные устройства, позволяет получать более полную и разностороннюю информацию о местоположении объекта~\cite{multsensors}. Синтез данных от этих разнообразных источников с помощью сложных алгоритмов фильтрации и слияния повышает точность и устойчивость к помехам определения местоположения~\cite{trends}. Кроме того, совместный анализ контекстной информации, такой как планировка помещения, расположение объектов инфраструктуры, модели движения людей, помогает точнее локализовать объекты.

Основными компонентами интегрированной системы определения местоположения в помещениях могут быть:

\begin{itemize}[label=---]
    \item сенсоры Wi-Fi и Bluetooth~\cite{blwifi} --- определение местоположения на основе силы и направления сигналов беспроводных сетей и устройств;
    \item инерциальные измерительные блоки (IMU)~\cite{imu} --- акселерометры, гироскопы и магнитометры для отслеживания перемещений и ориентации объектов;
    \item сенсоры видеонаблюдения~\cite{compvision} --- камеры и алгоритмы компьютерного зрения для визуального определения местоположения;
    \item радиочастотная идентификация (RFID)~\cite{basebook} --- метки и считыватели для определения местоположения объектов с метками;
    \item ультразвуковые датчики~\cite{shielding} --- определение местоположения на основе времени распространения ультразвуковых сигналов.
\end{itemize}

Данные от этих и других сенсоров обрабатываются с помощью методов обработки сигналов, фильтрации, слияния данных и алгоритмов машинного обучения для повышения точности и надежности определения местоположения.

Основными преимуществами технологии интегрированной системы определения местоположения в помещениях являются~\cite{multsensors}:

\begin{itemize}[label=---]
    \item высокая точность --- комбинирование данных от различных сенсоров позволяет достичь точности определения местоположения в пределах нескольких метров, что значительно превосходит возможности GPS в помещениях;
    \item надежность --- избыточность сенсоров и данных повышает отказоустойчивость системы, снижая вероятность сбоев и потери сигнала;
    \item масштабируемость --- модульная архитектура таких систем позволяет легко наращивать количество сенсоров и расширять зону покрытия для обслуживания объектов любого размера;
    \item гибкость --- возможность адаптации к различным условиям и конфигурациям зданий, а также интеграция с другими системами делает технологию универсальной и легко настраиваемой.
\end{itemize}

Несмотря на множество преимуществ, интегрированные системы геопозиционирования в помещениях также имеют ряд недостатков~\cite{multsensors}:

\begin{itemize}[label=---]
    \item высокая стоимость --- развертывание и обслуживание таких систем требует значительных капитальных и эксплуатационных затрат, связанных с установкой и обслуживанием множества сенсоров, вычислительной инфраструктуры и программного обеспечения;
    \item сложность внедрения --- интеграция разнородных сенсоров и данных, настройка алгоритмов и калибровка системы представляют собой трудоемкий и сложный процесс, требующий высокой квалификации специалистов;
    \item ограниченность географического покрытия --- такие системы, как правило, ограничены конкретными зданиями или объектами и не обеспечивают непрерывное отслеживание перемещений за их пределами;
    \item чувствительность к окружающей среде --- работа системы может нарушаться из-за электромагнитных помех, изменений в инфраструктуре здания, присутствия людей и других факторов, влияющих на распространение сигналов сенсоров.
\end{itemize}

% Несмотря на эти недостатки, интегрированные системы геопозиционирования в помещениях продолжают развиваться и находить все более широкое применение благодаря своей высокой точности, надежности и универсальности. Дальнейшее совершенствование технологий, снижение стоимости и повышение безопасности этих систем будут способствовать их более широкому внедрению в различных сферах.

\subsection{Облачные вычисления и машинное обучение}

Применение облачных вычислений~\cite{cloudcomp} и технологий машинного обучения~\cite{mltech} являются вспомогательными технологиями, позволяющими улучшить показатели других методов, открывая новые перспективы для систем геопозиционирования в помещениях. Облачные платформы обеспечивают вычислительные ресурсы, необходимые для обработки больших объемов сенсорных данных и реализации сложных алгоритмов локализации~\cite{cloudML}. Методы машинного обучения, такие как нейронные сети, позволяют автоматически извлекать скрытые закономерности в данных и повышать точность определения местоположения~\cite{mltech}. Кроме того, облачные решения упрощают интеграцию с другими системами и сервисами, расширяя возможности практического применения технологий геопозиционирования.

Основными преимуществами использования облачных вычислений и машинного обучения в системах геопозиционирования в помещениях являются~\cite{cloudML}:

\begin{itemize}[label=---]
    \item повышение точности и надежности --- облачные вычислительные мощности и алгоритмы машинного обучения позволяют обрабатывать большие объемы данных, выявлять сложные закономерности и улучшать точность определения местоположения;
    \item повышение эффективности --- автоматизация анализа данных, прогнозирование и поддержка принятия решений позволяют повысить эффективность использования ресурсов, оптимизировать процессы и снизить затраты;
    \item гибкость и масштабируемость --- облачная архитектура обеспечивает легкое масштабирование системы по мере роста количества сенсоров и пользователей, а также возможность быстрого развертывания и настройки;
    \item сокращение затрат --- отсутствие необходимости в локальной вычислительной инфраструктуре и передача вычислений в облако позволяют значительно снизить капитальные и эксплуатационные расходы;
    \item централизованное управление --- облачные технологии позволяют централизованно управлять системами геопозиционирования, удаленно настраивать алгоритмы и обновлять программное обеспечение;
    \item интеграция и взаимодействие --- облачные платформы упрощают интеграцию с другими корпоративными системами, обеспечивая единую и согласованную экосистему.
\end{itemize}

Несмотря на множество преимуществ, использование облачных вычислений и машинного обучения в системах геопозиционирования в помещениях также имеет ряд недостатков~\cite{cloudML}:

\begin{itemize}[label=---]
    \item зависимость от интернет-соединения --- стабильность и производительность системы напрямую зависит от качества интернет-соединения, что может быть проблематично в некоторых зданиях или удаленных объектах;
    \item сложность внедрения --- интеграция облачных технологий и настройка алгоритмов машинного обучения является более сложным и трудоемким процессом по сравнению с традиционными локальными системами;
    \item дополнительные расходы --- использование облачных вычислений и программного обеспечения может привести к дополнительным ежемесячным или ежегодным расходам, которые необходимо учитывать при бюджетировании;
    \item ограничения по обновлению --- обновление и доработка программного обеспечения, развернутого на облачной платформе, может быть ограничено возможностями поставщика облачных услуг;
    \item зависимость от поставщика облачных услуг --- использование облачных сервисов создает определенную зависимость от поставщика, что может усложнить переход на другие платформы в будущем.
\end{itemize}

\subsection{Выводы}

Использование альтернативных радиотехнологий, таких как WiFi обладает рядом ключевых преимуществ, включая широкую доступность инфраструктуры, достаточную точность, высокую отказоустойчивость и энергоэффективность. Хотя существуют определенные ограничения, связанные с многолучевым распространением сигналов и необходимостью предварительной калибровки, дальнейшее развитие методов обработки сигналов и картографирования помещений позволяет эффективно преодолевать эти проблемы. В совокупности, использование альтернативных радиотехнологий представляет собой конкурентоспособный и практичный подход к геопозиционированию в помещениях, который может найти широкое применение в различных отраслях и приложениях.

В работе будет развиваться подход с использованием WiFi технологий для геопозиционирования в помещениях.

\section{Существующие системы}

\subsection{Ekahau Real-Time Location System}

Ekahau Real-Time Location System (RTLS) --- это программно-аппаратное решение для определения местоположения объектов в помещениях в режиме реального времени. Система состоит из нескольких компонентов, обеспечивающих сбор, обработку и визуализацию данных о местоположении объектов.

Основными компонентами Ekahau RTLS являются:

\begin{enumerate}
    \item Метки Ekahau (Ekahau tags) --- это небольшие беспроводные устройства, прикрепляемые к отслеживаемым объектам. Метки периодически передают сигналы, содержащие информацию об их местоположении, используя радиочастотную связь.
    \item Точки доступа Ekahau (Ekahau access points) --- это устройства, принимающие радиочастотные сигналы от меток и передающие их на сервер обработки данных. Точки доступа размещаются по всей области, для которой требуется определение местоположения.
    \item Ekahau Positioning Engine --- программное обеспечение, выполняющее аналитическую обработку данных о радиочастотных сигналах, получаемых от точек доступа, и определяющее местоположение объектов, оснащенных метками.
    \item Ekahau Site Survey --- программное обеспечение для создания карт помещений и планирования развертывания системы Ekahau RTLS.
    \item Ekahau Vision --- программное обеспечение для визуализации местоположения объектов на планах помещений в режиме реального времени.
\end{enumerate}

Принцип работы Ekahau RTLS основан на измерении силы принимаемого сигнала (RSSI) от меток, установленных на отслеживаемых объектах. Метки периодически передают сигналы, которые регистрируются точками доступа. Затем данные о RSSI передаются на сервер Ekahau Positioning Engine, который на основе алгоритмов трилатерации определяет координаты местоположения меток. Полученная информация о местоположении визуализируется в программном обеспечении Ekahau Vision.

Ekahau RTLS поддерживает работу со множеством меток одновременно, обеспечивая отслеживание перемещений персонала, оборудования, инвентаря и других объектов. Система способна работать в различных типах помещений, включая офисы, больницы, отели, промышленные объекты и другие.

Ekahau RTLS предоставляет широкие возможности по интеграции с другими системами, такими как системы управления зданиями, медицинские информационные системы, бизнес-приложения и так далее. Это позволяет использовать данные о местоположении для решения широкого спектра задач, таких как оптимизация рабочих процессов, повышение эффективности использования ресурсов, обеспечение безопасности и многое другое.

Одним из преимуществ Ekahau RTLS является простота развертывания и настройки системы. Программное обеспечение Ekahau Site Survey позволяет легко создавать карты помещений и планировать расположение точек доступа, обеспечивая высокую точность определения местоположения. Кроме того, Ekahau RTLS отличается масштабируемостью, что позволяет адаптировать систему под потребности различных организаций, от небольших офисов до крупных предприятий.

\subsection{IndoorAtlas}

IndoorAtlas --- это платформа для определения местоположения в помещениях, основанная на использовании магнитного поля Земли. Система позволяет определять местоположение пользователей и объектов внутри зданий без необходимости развертывания дополнительной инфраструктуры, такой как Wi-Fi точки доступа или маячки Bluetooth.

Основными компонентами IndoorAtlas являются:

\begin{enumerate}
    \item Мобильное приложение или SDK --- программное обеспечение, установленное на мобильных устройствах пользователей. Оно использует встроенные в устройства датчики для измерения магнитного поля и передает эти данные на сервер IndoorAtlas.
    \item Сервер IndoorAtlas --- программное обеспечение, выполняющее обработку данных о магнитном поле и определение местоположения пользователей.
    \item Карты помещений --- цифровые планы зданий, загруженные в систему IndoorAtlas и содержащие информацию о структуре и особенностях магнитного поля внутри помещений.
\end{enumerate}

Принцип работы IndoorAtlas основан на том, что каждое здание имеет уникальный магнитный <<отпечаток>>, который определяется особенностями его строительных материалов, размещением металлических конструкций, электропроводки и другими факторами. Система IndoorAtlas использует датчики магнитного поля, встроенные в мобильные устройства пользователей, для измерения этого "отпечатка" и сравнения его с заранее созданными картами магнитного поля помещений.

Для создания карт магнитного поля специалисты IndoorAtlas проводят картографирование объектов, во время которого измеряют магнитное поле в различных точках помещений. Полученные данные загружаются в систему, формируя цифровые карты, которые впоследствии используются для определения местоположения пользователей.

Когда пользователь, использующий мобильное приложение или SDK IndoorAtlas, находится внутри здания, его устройство измеряет магнитное поле и передает данные на сервер. Сервер сравнивает полученные данные с картами магнитного поля, хранящимися в базе данных, и определяет местоположение пользователя в здании с точностью до нескольких метров.

Одним из главных преимуществ IndoorAtlas является то, что она не требует развертывания дополнительной инфраструктуры, такой как Wi-Fi точки доступа или Bluetooth маячки. Система использует уже существующие магнитные поля в зданиях, что значительно упрощает и удешевляет ее внедрение. Кроме того, IndoorAtlas обеспечивает высокую точность определения местоположения, сравнимую с другими технологиями позиционирования в помещениях.

Еще одним преимуществом IndoorAtlas является возможность ее интеграции с различными мобильными приложениями и сервисами, такими как навигация, управление доступом, отслеживание активов и другие. Это позволяет использовать данные о местоположении пользователей для решения широкого спектра задач.

В то же время, следует отметить, что точность определения местоположения в IndoorAtlas может варьироваться в зависимости от особенностей магнитного поля в конкретном здании. Кроме того, система может быть чувствительна к изменениям в магнитном поле, вызванным перемещением металлических объектов или работой электрооборудования. Поэтому для обеспечения высокой точности позиционирования требуется тщательное картографирование магнитного поля и регулярная калибровка системы.

\subsection{SenionLab}

SenionLab --- это платформа для определения местоположения в помещениях, использующая несколько технологий локализации, включая Bluetooth, Wi-Fi и магнитные поля.

Основными компонентами SenionLab являются:

\begin{enumerate}
    \item Сенсорные устройства --- небольшие беспроводные устройства, размещаемые в помещениях и оснащенные датчиками Bluetooth, Wi-Fi и магнитометрами. Эти устройства измеряют параметры окружающей среды и передают данные на сервер.
    \item Мобильное приложение или SDK --- программное обеспечение, устанавливаемое на мобильные устройства пользователей. Оно использует встроенные датчики для сбора данных и взаимодействует с серверным программным обеспечением SenionLab.
    \item Сервер SenionLab --- программное обеспечение, выполняющее обработку данных, полученных от сенсорных устройств и мобильных приложений, и определение местоположения пользователей.
    \item Карты помещений --- цифровые планы зданий, загруженные в систему SenionLab и содержащие информацию о расположении сенсорных устройств.
\end{enumerate}

Принцип работы SenionLab основан на комбинированном использовании нескольких технологий локализации:

\begin{enumerate}
    \item Bluetooth --- сенсорные устройства, оснащенные Bluetooth-приемниками, измеряют силу сигналов от мобильных устройств пользователей. Эти данные используются для определения приблизительного местоположения.
    \item Wi-Fi --- сенсорные устройства, оснащенные Wi-Fi-приемниками, измеряют параметры сигналов от точек доступа Wi-Fi, что также может быть использовано для локализации.
    \item Магнитное поле --- сенсорные устройства, оснащенные магнитометрами, измеряют магнитное поле в помещениях, позволяя определять местоположение пользователей, аналогично системе IndoorAtlas.
\end{enumerate}

Сервер SenionLab объединяет данные, полученные от различных источников, и применяет алгоритмы позиционирования, основанные на методах трилатерации и магнитного картографирования, для определения местоположения пользователей с высокой точностью.

Одним из преимуществ SenionLab является ее гибкость и возможность адаптации к различным условиям и требованиям. Система может работать как с использованием специализированных сенсорных устройств, так и с датчиками, встроенными в мобильные устройства пользователей. Кроме того, SenionLab поддерживает интеграцию с другими системами, такими как системы управления зданиями, навигационные приложения, системы безопасности и так далее.

Еще одной особенностью SenionLab является ее способность адаптироваться к изменениям в окружающей среде. Система постоянно собирает и анализирует данные от сенсорных устройств, что позволяет ей обновлять карты помещений и настраивать алгоритмы позиционирования для поддержания высокой точности определения местоположения.

Вместе с тем, внедрение SenionLab требует установки специализированных сенсорных устройств, что может увеличивать стоимость и сложность развертывания системы по сравнению с решениями, использующими только встроенные в мобильные устройства датчики, такие как IndoorAtlas. Кроме того, в некоторых случаях возможны сложности с интеграцией SenionLab с существующими системами и инфраструктурой организации.

\subsection{Анализ существующих систем и решений}

Ekahau RTLS, IndoorAtlas и SenionLab являются представителями различных подходов к определению местоположения в помещениях, каждый из которых имеет свои особенности и преимущества. Ekahau RTLS использует радиочастотную технологию и специализированные метки, обеспечивая высокую точность и масштабируемость. IndoorAtlas полагается на магнитное поле Земли, не требуя развертывания дополнительной инфраструктуры. SenionLab комбинирует несколько технологий локализации, обеспечивая гибкость и адаптивность.

\begin{table}[H]
    \caption{Анализ существующих систем позиционирования в помещении}
    \label{tab:es}
    \centering
    \begin{tabular}{|m{7 em}|m{8em}|m{8em}|m{8em}|}
        \hline
        Характеристика & Ekahau RTLS & IndoorAtlas & SenionLab \\
        \hline
        Технология & Wi-Fi, Bluetooth & Магнитные датчики, акселерометры & Wi-Fi, Bluetooth, магнитные датчики \\
        \hline
        Точность позиционирования & 1-3 метра & 2-5 метров & 1-3 метра \\
        \hline
        Требования к инфраструктуре & Доступ к точкам доступа Wi-Fi & Магнитные карты помещений & Сеть точек доступа, Bluetooth-маяки \\
        \hline
        Использование & Отслеживание перемещений персонала, оборудования, активов & Навигация, отслеживание положения & Отслеживание перемещений, навигация, поиск объектов \\
        \hline
        Область применения & Больницы, офисы, логистические центры & Торговые центры, музеи, аэропорты & Предприятия, склады, логистика \\
        \hline
      \end{tabular}
\end{table}

\subsection{Выводы}

Представленные в таблице \ref{tab:es} системы зависят от специфической инфраструктуры. Каждая система требует определенной инфраструктуры: точек доступа Wi-Fi, магнитных карт помещений или Bluetooth-маяков. Это может усложнять внедрение и увеличивать затраты на внедрение системы, особенно в уже существующих зданиях.

Для упрощения внедрения и изучения систем геопозиционирования в помещениях, необходимо предоставить пользователю возможность оценить влияние положения опорных точек и роутеров на эффективность позиционирования в помещениях.

\clearpage

\section{Анализ существующих методов}

Современные технологии дают возможность определять местоположение в открытой местности с помощью таких систем, как ГЛОНАСС и GPS. Представленные системы повсеместно используются, так как им необходим только один получатель (например, телефон) для определения местоположения~\cite{trends}. Однако сигналы от спутников GPS и ГЛОНАСС не могут проникать в помещения, чтобы точно определять местоположение объектов в закрытой местности, так как между спутником и объектом, местоположение которого необходимо определить, не должно быть препятствий~\cite{IGS}. 

В случае необходимости определения позиции объекта в помещении, применяют следующие технологии геопозиционирования: инерциальное~\cite{inertial}, Bluetooth~\cite{bluetooth}, WiFi~\cite{wifi}, ультразвуковое~\cite{ultrasound}, видимого освещения~\cite{light}. Наиболее популярными являются методы, использующие механизмы на основе WiFi~\cite{trends}.

Исторически выделяют три категории методов определения местоположения с применением устройств, генерирующих радиосигналы~\cite{wlan}:

\begin{itemize}[label=---]
    \item методы, применяющие AOA и связанный с ним DOA;
    \item методы, применяющие TOA и связанный с ним TDOA;
    \item методы, использующие опорные точки сигнала.
\end{itemize}

\subsection{Методы, основанные на AOA}

При рассмотрении методов, основным рабочим параметром которых является угол прибытия сигнала между принятой волной и некоторым направлением (ориентацией), необходимый угол измеряется как минимум от двух точек доступа~\cite{aoa}. 

\subsubsection{Алгоритм ангуляции}

На рисунке \ref{img:aoa} представлена простейшая LPS, состоящая из двух точек доступа и агента. Для определения местоположения источника сигнала (агента), необходимо задать координаты точек доступа --- $(x_1, y_1)$ и $(x_2, y_2)$ для первой и второй точки доступа соответственно. Искомые параметры --- координаты $(x_3, y_3)$ агента определяются по формулам (\ref{eq:aoa-x}) и (\ref{eq:aoa-y}).

\includeimage
    {aoa}
    {f}
    {H}
    {0.6\linewidth}
    {Принцип работы методов, основанных на AOA}

\begin{equation}
    x_3 = \frac{y_1 - y_2 + \frac{x_1}{tg\theta_1} + \frac{x_2}{tg\theta_2}}{\frac{1}{tg\theta_1} + \frac{1}{tg\theta_2}}
    \label{eq:aoa-x}
\end{equation}

\begin{equation}
    y_3 = y_1 - \frac{1}{tg\theta_1}\left(x_1 - \frac{y_1 - y_2 + \frac{x_1}{tg\theta_1} + \frac{x_2}{tg\theta_2}}{\frac{1}{tg\theta_1} + \frac{1}{tg\theta_2}}\right)
    \label{eq:aoa-y}
\end{equation}

Основной проблемой представленного метода является зависимость точности определения углов $\theta_1$ и $\theta_2$, необходимых для вычисления координат агента, от используемого оборудования, поэтому рассмотренный метод не является самым доступным~\cite{aoa-modern}. Также метод чувствителен к помехам и отражениям сигнала, что может привести к неточностям в определении местоположения, кроме того, точность этого метода снижается с увеличением дистанции между мобильным устройством и точками доступа. Это делает его менее надежным в больших помещениях или при наличии преград, таких как стены или перегородки~\cite{aoa}.

\clearpage

\subsection{Методы, основанные на TOA}

Методы, использующие для определения местоположения агента параметр TOA полагаются на время, за которое электромагнитная волна от агента достигает точки доступа~\cite{trends}. Метод требует строгой синхронизации часов, но отличается низкой стоимостью и низким потреблением энергии~\cite{toa}.

На рисунке \ref{img:toa} представлен пример LPS с использованием TOA.

\includeimage
    {toa}
    {f}
    {H}
    {0.6\linewidth}
    {Принцип работы методов, основанных на TOA}

\subsubsection{Алгоритм латерации}

Пусть $(x, y)$ --- неизвестные координаты агента, а $(x_i, y_i),~i \in [1, 3]$ --- координаты $i$-й точки доступа. Расстояние $R_i$ между точкой доступа и агентом определяется по формуле (\ref{eq:toa-r}):

\begin{equation}
    R_i = \sqrt{(x - x_i)^2 + (y - y_i)^2}.
    \label{eq:toa-r}
\end{equation}

Без ограничения общности, положим, что агент испускает сигнал во время $\tau_0 = 0$, а $i$-я точка доступа принимает сигнал во время $\tau_i$, где $\tau_i$ --- TOA. Тогда $R_i$ рассчитывается по формуле (\ref{eq:toa-t}):

\begin{equation}
    R_i = c \cdot \tau_i,
    \label{eq:toa-t}
\end{equation}
где $c$ --- скорость света. В таком случае позиция агента может быть определена как пересечение окружностей, с радиусами $R_1,~R_2,~R_3$, в условиях отсутствия помех~\cite{toa}.

Недостатком метода является ограниченная точность определения местоположения из-за влияния помех и отражений сигнала~\cite{toa}, когда сигнал от точки доступа распространяется через различные препятствия, могут возникать задержки и искажения, что усложняет точное определение времени прибытия сигнала и, следовательно, точности геопозиционирования.

Другим недостатком является ограничения аппаратного обеспечения, так как большинство коммерческих смартфонов не предназначены для точного измерения времени прибытия сигналов Wi-Fi --- у них отсутствует необходимое аппаратное обеспечение для выполнения таких измерений с высокой точностью~\cite{urban-info}.

\subsection{Методы, использующие опорные точки сигнала}

Методы, использующие опорные точки сигнала, также называемые методами сопоставления с образцом, представляют собой алгоритмы, построенные на анализе RSS. Сопоставление с образцом обычно происходит в два этапа: калибровка и геопозиционирование~\cite{intro}.

На рисунке \ref{img:falg} изображены две стадии работы метода опорных точек сигнала.

\includeimage
    {falg}
    {f}
    {H}
    {\linewidth}
    {Принцип работы методов, основанных на опорных точках сигнала}

В стадии калибровки (автономной фазы измерений) пространственно-временные RSS-данные от каждой точки доступа объединяются в общую базу данных и сохраняются как координаты опорных точек.

Местоположение агента может быть определено на стадии геопозиционирования с помощью измерения уровня сигналов от точек доступа и поиска наиболее точных совпадений в базе данных, с привязкой к точке на карте, где измерены такие же уровни сигналов.

Недостатками метода являются трудозатратный процесс конфигурации базы данных, необходимость переконфигурации при изменении ландшафта, а также высокая вычислительная сложность --- $O(N\cdot M)$, где $N$~--- количество записей в базе данных (количество опорных точек), а $M$~---~количество точек доступа $(M \geq 3)$~\cite{fingerprint}.

Методы геопозиционирования, основанные на опорных точках сигнала, используют один из двух подходов: детерминированный и стохастический~\cite{intro}.

\subsubsection{Детерминированный подход}

Детерминированный подход полагается на измерение уровня сигнала от каждой точки доступа и на основании этого определяет местоположение агента. Методы, использующие упомянутый подход, сравнивают уровни сигнала, полученные от агента с уровнями сигнала опорных точек, хранящихся в базе данных, оперируя Евклидовым расстоянием для определение ближайших соседей~\cite{fingerprint}.

\subsubsection{Алгоритм k-ближайших соседей}

Алгоритм k-ближайших соседей~\cite{knn} предполагает использование набора данных, полученного на стадии калибровки в методе опорных точек сигнала. Сам алгоритм работает следующим образом: для агента определяются k ближайших опорных точек из базы данных с помощью расстояния от $i$-й опорной точки до агента, которое рассчитывается по формуле (\ref{eq:fp:knn})

\begin{equation}
    D_i = \left(
        \frac{\sum\limits_{j=1}^{m} (|rss_j - rss_{ij}|)^t}{m}
    \right)^{\frac{1}{t}},
    \label{eq:fp:knn}
\end{equation}
где $i = \overline{1, n}$, $D_i$ --- манхэттенское расстояние от $i$-й опорной точки до агента (при $t = 1$) или евклидово расстояние (при $t = 2$), $m$ --- количество точек доступа, $rss_j$ --- уровень сигнала от агента до $j$-й точки доступа, $rss_{ij}$ --- уровень сигнала от $i$-й опорной точки до $j$-й точки доступа.

После выбора k ближайших к агенту опорных точек, местоположение агента может быть вычислено по формуле (\ref{eq:fp:location})

\begin{equation}
    (\hat{x}, \hat{y}) = \frac{\sum\limits_{i=1}^{k}(x_i, y_i)}{k},
    \label{eq:fp:location}
\end{equation}
где $(\hat{x}, \hat{y})$ --- координаты агента, $(x_i, y_i)$ --- координаты $i$-й опорной точки.

Ключевым преимуществом алгоритма k-ближайших соседей является эффективность при достаточном объеме данных, возможность использовать алгоритм в разнообразных сценариях, не имея предположений о распределении данных~\cite{trends}.

Недостатками алгоритма является высокая вычислительная сложность при большом объеме данных, а также требования к опорным точкам сигналов: на вход необходимо подать подготовленный набор данных, который минимизирует ошибки при аппроксимации местоположения агента в помещении~\cite{android-analysis}.

\subsubsection{Алгоритм взвешенных k-ближайших соседей}

Алгоритм взвешенных k-ближайших соседей~\cite{gansemer2010improved} заключается в присваивании весовых коэффициентов координатам различных опорных точек, которые обычно устанавливаются как обратное значение евклидова расстояния между опорной точкой и агентом~\cite{wknn}. Данный алгоритм поддается улучшениям, что показывает множество исследований, например, модификация алгоритма, которая заключается в высчитывания весовых коэффициентов на основе модели распространения сигнала, благодаря выявленной нелинейной связи между уровнем сигнала и физическим расстоянием~\cite{mwknn}.

\subsubsection{Алгоритмы на основе нейронных сетей}

Алгоритмы геопозиционирования на основе метода опорных точек, использующие нейронные сети, анализируют наборы векторов, описывающих распространение сигналов в опорных точках, используя эти данные для обучения~\cite{neural-networks}. 

Нейронная сеть может быть спроектирована как многослойный перцептрон, способный обрабатывать входные векторы отпечатков и выводить прогнозируемое местоположение. Обучение модели происходит путем подачи входных данных (пар местоположений опорных точек и векторов уровней сигналов в этих точках).

После завершения обучения модель можно использовать для геопозиционирования в реальном времени. При поступлении нового отпечатка сигнала WiFi нейронная сеть обрабатывает его и предсказывает местоположение, основываясь на заранее выученных закономерностях~\cite{trends}.

Преимущества подхода включают высокую точность определения местоположения в помещениях, а также возможность обучения модели на конкретных объектах, что позволяет улучшить качество геопозиционирования для конкретного помещения или даже отдельной зоны~\cite{neural-networks}.

Основные недостатки алгоритмов на основе нейронных сетей включают в себя следующие факторы: изменчивость среды может сильно влиять на точность геопозиционирования; для реализации геопозиционирования с помощью нейронных сетей необходимы дополнительные вычислительные ресурсы при большом количестве точек доступа и опорных точек.

\subsubsection{Стохастический подход}

Стохастический подход предполагает использование функции условного распределения для неизвестных параметров (координат агента)

Одной из ключевых особенностей стохастических алгоритмов является их способность учитывать случайные шумы и изменения в окружающей среде. Это означает, что данные, полученные с помощью метода опорных точек, могут быть подвержены различным искажениям, таким как влияние стен и преград, а также изменения внутренних условий помещения. Стохастические алгоритмы способны адаптироваться к таким условиям и проводить статистическую обработку данных для определения местоположения с учетом неопределенности~\cite{trends}.

Использование стохастических алгоритмов требует предварительной калибровки с помощью точных измерений сигналов WiFi в различных точках помещения~\cite{lin2021probabilistic}. Это может быть сложным и затратным процессом, требующим значительных ресурсов, особенно для больших помещений или для помещений с переменной конфигурацией.

\subsubsection{Скрытая Марковская модель}

% Скрытая модель Маркова состоит из двух стохастических процессов. Первый стохастический процесс представляет собой марковскую цепь, которая характеризуется состояниями и вероятностями перехода между ними. Состояния цепи не наблюдаются напрямую, поэтому они и называются "скрытыми". Второй стохастический процесс генерирует вероятность излучения, связанную с скрытыми состояниями в каждый момент времени, в зависимости от условного распределения состояний.

% Марковская цепь --- это стохастический процесс, включающий набор $N$ состояний с вероятностными переходами между состояниями. Вероятностное распределение переходов состояний обычно представляется как матрица переходов марковской цепи. Если цепь в настоящее время находится в состоянии $S_i$, то она переходит в состояние $S_j$ на следующем шаге с вероятностью, обозначаемой как $a_{ij}$. Матрица $A$ - это матрица вероятностей переходов размером $N \times N$, каждый элемент $a_{ij}$ известен как вероятность перехода и зависит только от текущего состояния $S_i$. Марковским называется процесс, обладающий следующим свойством: для каждого момента времени $t_0$ вероятность любого состояния системы в будущем зависит только от состояния системы в настоящем и не зависит от того, когда и каким образом система пришла в это состояние.

Скрытая Марковская модель --- это статистическая модель, в которой система моделируется как Марковский процесс с ненаблюдаемыми состояниями~\cite{mor2021systematic}. Её можно использовать для решения задачи геопозиционирования в помещении, так как она подходит для обработки временных рядов, которые представляют собой информацию о полученных агентом и опорными точками уровнями сигналов.

Рассмотрим помещение, разделённое на несколько областей (состояний модели), таких как комнаты или коридоры. Будем считать, что местоположение объекта в конкретный момент времени --- это состояние модели. Для примера, в случае, если у объекта только два возможных местоположения, мы получаем модель с двумя состояниями.

Вектор уровней сигналов, служит <<наблюдением>> в контексте скрытой Марковской модели. При использовании метода опорных точек на этапе обучения собираются данные о силе сигнала WiFi в каждом из возможных состояний, создавая так называемую <<опорную точку>>. Задача в этом контексте --- предсказать скрытое состояние модели (текущее местоположение объекта) на основе набора измерений векторов сигналов.

Формулы для Скрытой Марковской модели можно записать с помощью переходных (\ref{eq:mark:trans}) и эмиссионных (\ref{eq:mark:emission}) вероятностей.

\begin{eqnarray}
    P(i, j) &=& P(S_{t+1}=j|S_t=i), \label{eq:mark:trans} \\
    B_j(k) &=& P(O_t=k|S_t=j), \label{eq:mark:emission}
\end{eqnarray}
формула (\ref{eq:mark:trans}) представляет собой вероятность того, что система перейдёт из состояния $i$ в состояние $j$, а формула (\ref{eq:mark:emission}) располагает вероятностью того, что при сигнале $k$, система окажется в состоянии $j$.

Основная задача в этом случае --- на основе наблюдаемых данных (уровней сигналов WiFi) определить фактическое местоположение. Это можно сделать с помощью алгоритма Витерби~\cite{dong2020sequential}, который вычисляет наиболее вероятную последовательность скрытых состояний, давших наблюдаемые результаты.

Таким образом, построив Скрытую Марковскую модель и использовав уровни сигналов WiFi, можно определить местоположение объекта в помещении. Важным преимуществом подхода является его способность учитывать историю перемещения объекта, что делает его более точным при определении местоположения объекта в динамике, однако алгоритм неустойчив к изменениям в среде и интерференциям, а также требует предварительной калибровки~\cite{rudic2020geometry}.

\clearpage

\subsection*{Выводы из подраздела}

В результате рассмотрения методов геопозиционирования в помещениях на основе Wi-Fi, можно сделать следующие выводы:

\begin{itemize}[label=---]
    \item метод на основе анализа времени прибытия сигнала обеспечивает возможность точного позиционирования агента, при условии синхронизации часов на устройствах, участвующих в геопозиционировании, однако является неустойчивым к изменениям в среде и преградам между точкой доступа и агентом;
    \item метод на основе анализа угла прибытия сигнала обеспечивает простую аппроксимацию местоположения агента на коротких расстояниях, но требует дополнительного оборудования для работы, а также может выдавать неверные результаты на больших расстояниях и при наличии помех и преград;
    \item метод опорных точек сигнала поддается модификациям, что дает настроить геопозиционирование в конкретных помещениях, учитывая возможные помехи и преграды между точками доступа, однако неустойчив к изменениям в среде и требует предварительной калибровки перед использованием. 
\end{itemize}

\clearpage

% \subsection{Методы геопозиционирования на основе Bluetooth}

% Bluetooth с низким энергопотреблением (BLE) появился в стандарте Bluetooth 4.0 и рассчитан на беспроводную передачу данных на коротком расстоянии~\cite{bluetooth}. Исходя из этих данных, ожидаемые ошибки при использовании методов геопозиционирования, основанных на Bluetooth гораздо меньше, в сравнении с методами геопозиционирования на основе WiFi~\cite{blwifi}.

% % В то же время, по сравнению с методами определения местоположения агента внутри помещений, основанных на использовании точек доступа WiFi, системы геопозиционирования, основанные на BLE всегда требуют дополнительного оборудования, благодаря чему могут достигнуть более высокой точности~\cite{android-analysis}.

% Методы геопозиционирования с помощью WiFi и Bluetooth включают в себя измерение сигналов от ближайших точек доступа (для Wi-Fi) или маяков (для BLE), а затем использование этой информации для вычисления координат устройства. Это предоставляет возможность использовать все вышеописанные методы геопозиционирования в помещениях и для BLE. Однако есть несколько ключевых различий между этими двумя технологиями.

% Главным различием служит сама технология передачи данных. Wi-Fi использует стандарт IEEE 802.11~\cite{ieee802} для передачи данных, в то время как BLE использует более энергоэффективный протокол передачи данных. Это позволяет BLE использовать меньше энергии и обеспечивает длительное время работы устройства без подзарядки~\cite{bluetooth}.

% Еще одним различием является диапазон работы. Wi-Fi обычно имеет больший диапазон работы, чем BLE, что делает его более подходящим для позиционирования на больших расстояниях~\cite{wlan}. Однако BLE обычно обеспечивает более точное позиционирование на небольших расстояниях благодаря использованию маяков и возможности определения расстояния до устройства с точностью до нескольких сантиметров~\cite{android-analysis}.

% Кроме того, BLE обычно имеет более низкую задержку передачи данных по сравнению с Wi-Fi \cite{android-analysis}, что делает его более подходящим для реального времени позиционирования, однако для геопозиционирования в больших зданиях необходимо приобрести множество маяков, в связи с небольшим диапазоном работы BLE.

% Таким образом, обе технологии имеют свои преимущества и недостатки, и выбор между ними зависит от конкретных требований проекта, использующего геопозиционирование.

% \clearpage

% \subsection*{Выводы из подраздела}

% Методы геопозиционирования в помещениях с использованием Bluetooth являются одними из наиболее эффективных средств определения местоположения агента внутри здания, однако не лишены недостатков.

% Для работы методов геопозиционирования на основе BLE используются те же методы, что и для Wi-Fi, в результате чего сохраняются их преимущества и недостатки. Однако BLE является более энергоэффективной технологией, а также позволяет определить расстояние до устройства с точностью до нескольких сантиметров~\cite{bluetooth}. 

% Основными недостатками методов, использующих BLE является низкий диапазон работы, что делает их более дорогим решением, в сравнении с использованием методов на основе Wi-Fi, в связи с тем, что необходимо приобретение множества маяков.

\clearpage

\subsection{Классификация методов геопозиционирования на основе WiFi}

На рисунке \ref{img:classification} приведена классификация рассмотренных методов геопозиционирования в помещениях на основе Wi-Fi.

\includeimage
    {classification}
    {f}
    {h}
    {\linewidth}
    {Классификация методов геопозиционирования в помещении}

\clearpage

В таблице \ref{table:classification} приведено сравнение методов геопозиционирования в помещении.

\begin{table}[ht]
    \caption{Классификация алгоритмов геопозиционирования в помещении}
    \begin{tabular}{|m{3cm}|m{3.5cm}|m{3cm}|m{5.5cm}|}
        \hline
        Метод & Используемая информация & Измерения & Особенности \\
        \hline
        \hline
        Анализа времени прибытия сигналов & геометрические параметры & время прибытия сигнала & ограничения аппаратного обеспечения; сильное влияние помех и отражений сигнала на результат \\
        \hline
        Анализа угла прибытия сигналов & геометрические параметры & углы прибытия сигнала & необходимо дополнительное оборудование; чувствителен к помехам и отражениям сигнала; ненадежен на больших расстояниях \\
        \hline
        Опорных точек сигнала & статистические и эмпирические результаты & уровень сигнала & отсутствие необходимости в дополнительном оборудовании для помещений, в которых настроена сеть Wi-Fi; затруднительны для использования в часто изменяющихся пространствах; требуют предварительной калибровки \\
        \hline
    \end{tabular}
    \label{table:classification}
\end{table}

Исходя из классификации видно, что наиболее гибким и эффективным методом определения местоположения является использование метода, основанного на опорных точках сигнала. Это связано с тем, что для конкретного помещения можно занести в базу данных, необходимую для работы алгоритма, специально расположенные опорные точки, которые будут учитывать помехи и отражения сигнала, давая более точные данные для определенного набора помещений~\cite{bag}. Также следует отметить, что алгоритмы, работающие на основе метода опорных точек, не требуют дополнительных закупок со стороны организации, планирующей использовать его в своих помещениях, если в них настроена сеть Wi-Fi с более, чем тремя точками доступа.

\subsection{Выводы}

Метод опорных точек сигнала является одним из наиболее подходящих для геопозиционирования внутри помещений с использованием WiFi. В сравнении с другими методами, он имеет ряд преимуществ, которые делают его привлекательным выбором в таких условиях.

Во-первых, метод опорных точек сигнала не требует дополнительного специализированного оборудования, кроме уже установленной WiFi-инфраструктуры в помещении. Это делает его более экономичным и простым в реализации, в отличие от методов анализа времени прибытия или угла прибытия сигналов, которым необходимы специальные датчики и антенны.

Во-вторых, метод опорных точек сигнала основан на статистических и эмпирических данных об уровне сигнала в различных точках помещения. Это делает его более устойчивым к помехам и отражениям сигнала, которые часто встречаются в закрытых пространствах и могут существенно влиять на результаты других методов, таких как анализ времени или угла прибытия сигналов.

Однако следует учитывать, что применение метода опорных точек сигнала требует предварительной калибровки - построения карты опорных точек в помещении. Это может быть трудоемким процессом, особенно в случае часто изменяющихся пространств. Тем не менее, при наличии стабильной WiFi-инфраструктуры и возможности проведения предварительной калибровки, метод опорных точек сигнала представляется наиболее подходящим выбором для геопозиционирования внутри помещений.

\section{Оптимизация положения опорных точек сигнала}

Одним из ключевых аспектов задачи геопозиционирования с использованием метода опорных точек является оптимальное расположение опорных точек, служащих для определения положения объектов. Добиться оптимального расположения опорных точек можно с помощью методов оптимизации, таких как генетические алгоритмы, алгоритмы роя частиц и имитации отжига.

\subsection{Генетические алгоритмы}

Генетические алгоритмы (ГА) основаны на принципах естественного отбора и генетики, где популяция особей эволюционирует для нахождения оптимального решения.
ГА могут быть эффективны при решении сложных многомодальных задач, однако они требуют тщательной настройки параметров, таких как размер популяции, вероятности скрещивания и мутации.

Сходимость ГА может быть медленной, особенно в задачах с большим числом переменных, что негативно сказывается на времени вычислений.

\subsection{Алгоритмы роя частиц}

Алгоритмы роя частиц (PSO) имитируют поведение стаи птиц или рыб, где каждая частица представляет собой возможное решение задачи.
PSO отличаются простотой реализации, быстрой сходимостью и хорошей масштабируемостью даже для задач с большим числом переменных.
В контексте расстановки опорных точек в помещении с использованием WiFi, PSO эффективно находит оптимальное расположение точек, учитывая такие факторы, как сила сигнала и покрытие области.

\subsection{Имитация отжига}

Алгоритм имитации отжига (Simulated Annealing, SA) моделирует процесс термического отжига в металлургии для поиска глобального минимума функции.
SA может быть эффективен при решении задач с большим количеством локальных минимумов, однако настройка параметров алгоритма, таких как температура и скорость охлаждения, требует тщательной подготовки.
Применительно к задаче расстановки опорных точек в помещении с использованием WiFi, SA может столкнуться с проблемами преждевременной сходимости в локальных минимумах, что негативно скажется на качестве решения.

\subsection{Выводы}

\begin{table}[H]
    \caption{Сравнительный анализ методов оптимизации}
    \centering
    \begin{tabular}{|m{5cm}|m{5cm}|m{5cm}|}
    \hline
    Метод & Преимущества & Недостатки \\
    \hline
    Генетические алгоритмы & - Эффективны для сложных многомодальных задач & - Требуют тщательной настройки параметров \\
    & & - Медленная сходимость, особенно для задач с большим числом переменных \\
    \hline
    Алгоритмы роя частиц & - Простота реализации & - Могут застревать в локальных минимумах \\
    & - Быстрая сходимость & \\
    & - Хорошая масштабируемость & \\
    & - Эффективны для задачи расстановки опорных точек в помещении с использованием WiFi & \\
    \hline
    Имитация отжига & - Эффективны при наличии большого числа локальных минимумов & - Требуют тщательной настройки параметров \\
    & & - Возможна преждевременная сходимость в локальных минимумах \\
    \hline
    \end{tabular}
\end{table}

Алгоритмы роя частиц являются предпочтительным методом оптимизации для решения задачи расстановки опорных точек в помещении с использованием WiFi сигналов. Их простота реализации, быстрая сходимость и хорошая масштабируемость делают их более эффективными по сравнению с генетическими алгоритмами и имитацией отжига в контексте решаемой задачи.

\section{Формализация постановки задачи}

Метод опорных точек сигнала для геопозиционирования в помещениях с использованием сигналов Wi-Fi основывается на сравнении измеренных характеристик сигнала Wi-Fi в произвольной точке помещения с заранее собранной базой данных опорных точек.

\subsection*{Входные данные}

\begin{enumerate}
    \item Топология комнаты в виде многоугольника.
    \item Список опорных точек, содержащий информацию о характеристиках сигналов Wi-Fi в каждой из опорных точек. Типичными характеристиками являются уровень принимаемого сигнала (RSSI) от различных точек доступа Wi-Fi.
    \item Текущие измерения характеристик сигналов Wi-Fi в произвольной точке помещения, для которой необходимо определить местоположение (в точке нахождения агента).
\end{enumerate}

\subsection*{Выходные данные}

\begin{enumerate}
    \item Оценка местоположения объекта в помещении, выраженная в координатах на карте.
    \item Координаты ближайшей опорной точки.
\end{enumerate}

Применение метода позволяет достичь точности геопозиционирования в помещениях на уровне нескольких метров при относительно невысокой стоимости развёртывания системы, поскольку он использует существующую инфраструктуру Wi-Fi.

\clearpage

На рисунке \ref{img:idef0} представлена формализация постановки задачи в виде IDEF0-диаграммы.

\includeimage
    {idef0}
    {f}
    {H}
    {\linewidth}
    {Формализация постановки задачи в виде IDEF0}
